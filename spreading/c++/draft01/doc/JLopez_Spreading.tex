\title{Numerical solution of a nonlinear thin-film equation:
\\
gravity-driven spreading of a thin film with surface tension over a horizontal substrate
}
\author{
        Joseph M. Barakat
}
\date{\today}
%the percentage makes comments
%this is the preamble where all the packages are loaded. I would always have these packages.
\documentclass[12pt]{article}
\usepackage[english]{babel}
\usepackage{graphicx}%this allows you to add figures
\usepackage{float}
\usepackage{amsmath}%this lets you write math
\usepackage{mathtools}
\usepackage{amssymb}%this lets you write math symbols
\usepackage{mathrsfs}
\numberwithin{equation}{section}
\usepackage{caption}%this lets you caption figure
\usepackage{wrapfig}
\usepackage{setspace}%this lets you control line spacing
\usepackage{siunitx}%this lets you make SI units
\usepackage[margin=1in]{geometry}
\usepackage[version=3]{mhchem}
\usepackage{csquotes}
\usepackage{cancel}
\usepackage{esint}
\usepackage{courier}
\usepackage{bm,upgreek}
\usepackage{braket}
%\usepackage{wasysym}
%\usepackage{breqn}
\usepackage{appendix}
\usepackage{stmaryrd}
\usepackage{hyperref}
\usepackage[final]{pdfpages}
\usepackage{bbm}
%\usepackage{calligra}
%\usepackage[T1]{fontenc}
\usepackage{perpage}
\usepackage{lscape}
\MakePerPage{footnote}

% Insert MATLAB and Mathematica  code into document
\usepackage{listings}
\lstloadlanguages{[5.2]Mathematica}
\usepackage{color} %red, green, blue, yellow, cyan, magenta, black, white
\definecolor{mygreen}{RGB}{28,172,0} % color values Red, Green, Blue
\definecolor{mylilas}{RGB}{170,55,241}

\DeclareMathOperator{\tr}{tr}
\DeclareMathOperator{\sech}{sech}
%
%\DeclareMathAlphabet{\mathcalligra}{T1}{calligra}{m}{n}
%\DeclareFontShape{T1}{calligra}{m}{n}{<->s*[1.7]callig15}{}

\begin{document}
\maketitle

\allowdisplaybreaks[1]

\renewcommand*{\thefootnote}{\fnsymbol{footnote}}

% Insert MATLAB script into text
\lstset{language=MATLAB,%
    %basicstyle=\color{red},
    breaklines=true,%
    morekeywords={matlab2tikz},
    keywordstyle=\color{blue},%
    morekeywords=[2]{1}, keywordstyle=[2]{\color{black}},
    identifierstyle=\color{black},%
    stringstyle=\color{mylilas},
    commentstyle=\color{mygreen},%
    showstringspaces=false,%without this there will be a symbol in the places where there is a space
    numbers=left,%
    numberstyle={\tiny \color{black}},% size of the numbers
    numbersep=9pt, % this defines how far the numbers are from the text
    emph=[1]{for,end,break},emphstyle=[1]\color{red}, %some words to emphasise
    %emph=[2]{word1,word2}, emphstyle=[2]{style},    
}

% To insert MATLAB script, type
%	\texttt{\lstinputlisting{[...].m}}

% To insert figure, type
%	\begin{figure}[h]
%	\begin{center}
%		\includegraphics[width=.45\textwidth]{[...].jpg}
%	\caption{...}
%	\label{fig:?}
%	\end{center}
%	\end{figure}


\section{References}

J Lopez, {\it Dynamics of wetting processes}, { Ph.D. Thesis (Carnegie Mellon University),} (1975)

\noindent
D  U von Rosenberg, {\it Methods for the Numerical Solution of Partial Differential Equations} (1969)

\section{Basic equations (dimensionAL form)}

\begin{align}
	\frac{\partial h}{\partial t} &=  \frac{1}{\sigma} \frac{\partial }{\partial \sigma}  \left( \sigma \frac{h^3}{3\mu} \frac{\partial P}{\partial \sigma} \right)
	\\
	P &
	= \rho g h - \frac{\gamma}{\sigma}\frac{\partial }{\partial \sigma} \left( \sigma \frac{\partial h}{\partial \sigma} \right)  + \frac{A}{6 \pi h^3}
	\\
	h(\sigma,0) &= h(\sigma)
	\\
	\frac{\partial h}{\partial \sigma} (0,t) &= 0
	\\
	\frac{\partial P}{\partial \sigma}(0,t) &=0
	\\
	h[r(t),t] &= h^*
	\\
	\frac{\partial h}{\partial \sigma}[r(t),t] &= -\alpha
	\\
	2 \pi \int_0^{r(t)} h(\sigma,t) \, \sigma \, \mathrm{d}\sigma &= \mit{\Omega}
\end{align}
where $t$ is time, $\sigma$ is the radial position $h$ is the film thickness, $P$ is the pressure, $\gamma$ is the surface tension, $\rho$ the density, $g$ the acceleration due to gravity, $\mu$ the viscosity, $A$ the Hamaker constant, $\alpha$ the contact slope, $h^*$ the contact height, $\Omega$ the drop volume, and $r$ the drop radius.
%  $k(h) = \rho g h^3/(3\mu)$ is the diffusivity, $\lambda = \sqrt{\gamma / (\rho g)}$ is the capillary length, and $\mit{\Omega}$ is the drop volume. 

\subsection{Reformulation of the free-boundary problem}

The boundary $\sigma = r(t)$ is free. Need to redefine the problem on a fixed domain.

 Define the modified coordinates:
 \[
 	\xi(\sigma, t) = \frac{\sigma}{r(t)}
 	,\qquad
 	\tau (t)
 	=
 	\frac{t}{[r(t)]^2}
  \] 
By the chain rule,
\[
	\frac{\partial}{\partial \sigma} = \frac{1}{r} \frac{\partial}{\partial \xi},
	\qquad
	\frac{\partial}{\partial t} 
	= \frac{1}{r^2} \left( 1 - \frac{2t}{r} \frac{\mathrm{d}r}{\mathrm{d}t} \right) \frac{\partial}{\partial \tau}
	= \frac{1}{r^2} \left( 1 - \frac{2\tau}{r} \frac{\partial r}{\partial \tau} + O[( {\partial r}/{\partial \tau})^2] \right) \frac{\partial}{\partial \tau},
\]
\[
	\int_0^{r(t)} ( ~\cdot ~) \, \sigma \, \mathrm{d}\sigma = r^2 \int_0^1 ( ~ \cdot ~) \, \xi \, \mathrm{d}\xi
\]
%where $\dot{r} = \mathrm{d} r / \mathrm{d} t = \mathrm{d} r / \mathrm{d} \tau - (2 \tau / r)(\mathrm{d} r / \mathrm{d} \tau )^2 + \cdots $. \
Thus, we obtain the modified problem:
\begin{align}
	d(\tau)\frac{\partial h}{\partial \tau} &= \frac{1}{\xi} \frac{\partial }{\partial \xi}  \left( \xi k(h) \frac{\partial p}{\partial \xi} \right)
	\\
	p &
	= h - \frac{\lambda^2}{r^2} \frac{1}{\xi}\frac{\partial }{\partial \xi} \left( \xi \frac{\partial h}{\partial \xi} \right) 
	+ \frac{\kappa}{h^3}
	\\
	h(\xi,0) &= h(\xi)
	\\
	\frac{\partial h}{\partial \xi} (0,\tau) &= 0
	\\
	\frac{\partial p}{\partial \xi}(0,\tau) &=0
	\\
	h(1,\tau) &= h^*
	\\
	\frac{\partial h}{\partial \xi}(1,\tau) &= - \alpha r
	\\
	2 \pi r^2 \int_0^{1} h(\xi,\tau) \, \xi \, \mathrm{d}\xi &= \mit{\Omega}
\end{align}
where
\[
	p = 
	\frac{P}{\rho g}, \quad
	\lambda^2 = \frac{\gamma}{\rho g},\quad
	\kappa = \frac{A}{6 \pi \rho g},
	\quad
	d(\tau)  = 1 - \frac{\mathrm{d}r^2}{\mathrm{d}t}\tau, \quad
	k(h) = \frac{\rho g h^3}{3\mu}
\]
We needed to transform the time derivative so that the spreading velocity appears explicitly in the equation. At each timestep, we update the spreading distance using the computed velocity.

Note that the integral constraint is used to solve for the free boundary $r(\tau)$.

\section{Finite difference analogy}

Discretize $\{ \xi_j\}$, $\{ \tau_n\}$, $j = 0, 1, \dots, J$, $n = 0, 1, \dots, N$ on a uniform grid, where $\xi_j = j \Delta \xi$ and $\tau_n = n \Delta \tau$. Shift $h$ and $p$ by 1/2 step in space, so that $h_{j,n} = h(\xi_{j+\frac{1}{2}},t_n)$ and $p_{j,n} = p(\xi_{j+\frac{1}{2}},t_n)$. This is useful for two reasons. Firstly, the BCs can be easily applied. Second, we avoid the singularity at $\xi = 0$.


For now, ignore the van der Waals term in the equation for the pressure (i.e. set $\kappa = 0$). Might find that we are unable to satisfy the right BC $h = 0$ at $\xi = 1$, but let's find out later...


\subsection{Boundary conditions}

\begin{align}
	\frac{h_{0,n} - h_{-1,n}}{\Delta \xi} &= 0
	\\
	\frac{p_{0,n} - p_{-1,n}}{\Delta \xi} &= 0
	\\
	\frac{h_{J-1,n} + h_{J,n}}{2} &= h^*
	\\
	\frac{h_{J,n} - h_{J-1,n}}{\Delta \xi} &= - \alpha r_n
	\\
	\pi r_n^2 \Delta \xi^2 \left[ Jh^* +   \sum_{j=1}^{J-1} j (h_{j-1,n} + h_{j,n}) \right]& = \mit{\Omega}
\end{align}
from which we obtain
\begin{align}
	 h_{-1,n} &=  h_{0,n}
	\\
	 p_{-1,n}&=  p_{0,n}
	\\
	 h_{J-1,n}&= h^* + \frac{\alpha r_{n} \Delta \xi}{2}
	\\
	h_{J,n}&=  h^*- \frac{\alpha r_{n} \Delta \xi}{2}
	\\
	 r_n &= \left[
	 	\frac{\mit{\Omega}}{\pi  \Delta \xi^2 \left[ Jh^* +   \sum_{j=1}^{J-1} j (h_{j-1,n} + h_{j,n}) \right]}
	 \right]^{\frac{1}{2}}
\end{align}
Thus we eliminate $h_{-1,n}$, $h_{J-1,n}$, and $h_{J,n}$ as unknowns.  




\subsection{Centered difference analogs for $2 \le j \le J-4$}

Centered difference equations at $\xi_{j+\frac{1}{2}}$, $\tau_{n+\frac{1}{2}}$ (note that $h_{j,n+\frac{1}{2}}$, $p_{j,n+\frac{1}{2}}$ corresponds to the physical point $\xi_{j+\frac{1}{2}}, \tau_{n+\frac{1}{2}}$):
\begin{align}
	d_{n+\frac{1}{2}}
	\frac{h_{j, n+1} - h_{j,n}}{\Delta \tau}
	&
	=
	\frac{1}{(j +\frac{1}{2})\Delta \xi^2}
	\left[
		(j + 1) \Delta \xi k_{j+\frac{1}{2},n+\frac{1}{2}} \left( \frac{p_{j+1,n+\frac{1}{2}} - p_{j,n+\frac{1}{2}}}{\Delta \xi} \right)
	\right.
	\nonumber \\ &\qquad\qquad\qquad\qquad
	\left.
		-
		j \Delta \xi k_{j-\frac{1}{2},n+\frac{1}{2}} \left( \frac{p_{j,n+\frac{1}{2}} - p_{j-1,n+\frac{1}{2}}}{\Delta \xi} \right)
	\right]
	\\
	p_{j,n+\frac{1}{2}}
	&=
	h_{j,n+\frac{1}{2}}
	-
		\frac{\lambda^2}{(j + \tfrac{1}{2})r^2_{n+\frac{1}{2}} \Delta \xi^2}
	\left[
		(j+1)\Delta \xi \left( \frac{h_{j+1,n+\frac{1}{2}} - h_{j,n+\frac{1}{2}}}{\Delta \xi} \right)
	\right.
	\nonumber \\ &\qquad\qquad\qquad\qquad\qquad\qquad\qquad
	\left.	
		+
		j \Delta \xi \left( \frac{h_{j,n+\frac{1}{2}} - h_{j-1,n+\frac{1}{2}}}{\Delta \xi} \right)
	\right]
	\\
	d_{n+\frac{1}{2}}
	&=
	1 - 2 r_{n+\frac{1}{2}} \left( \frac{\mathrm{d}r}{\mathrm{d}t}\right)_{n+\frac{1}{2}} (n + \tfrac{1}{2}) \Delta \tau
	\\
	k_{j+\frac{1}{2},n+\frac{1}{2}}
	&=
	\frac{\rho g}{3 \mu} \left( \frac{h_{j,n+\frac{1}{2}}+h_{j+1,n+\frac{1}{2}}}{2} \right)^3
	\\
	k_{j-\frac{1}{2},n+\frac{1}{2}}
	&=
	\frac{\rho g}{3 \mu} \left( \frac{h_{j-1,n+\frac{1}{2}}+h_{j,n+\frac{1}{2}}}{2} \right)^3
\end{align}
Obviously, we can recombine some terms:
\begin{align}
	h_{j, n+1} - h_{j,n}
	&
	=
	\frac{ \Delta \tau}{(j +\frac{1}{2})d_{n+\frac{1}{2}}\Delta \xi^2}
	\left\{
		jk_{j-\frac{1}{2}} p_{j-1}
		-
		[
			jk_{j-\frac{1}{2}}
			+
			(j+1)k_{j+\frac{1}{2}}
		]p_{j}
		+
		(j+1)k_{j+\frac{1}{2}} p_{j+1}
	\right\}_{n+\frac{1}{2}}
	\nonumber \\
	p_{j}(\tau)
	&=
	h_{j}
	-
		\frac{\lambda^2}{(j + \tfrac{1}{2})r^2 \Delta \xi^2}
	\left[
		j h_{j-1} - (2j+1) h_j + (j+1) h_{j+1}
	\right]
\end{align}
Now, we replace $\vec{p}_{n+\frac{1}{2}}$ with the average of the values of $t_n$ and $t_{n+1}$,
\[
	p_{j,n+\frac{1}{2}} = \frac{p_{j,n+1} + p_{j,n}}{2}
\]
so that we obtain a difference equation of Crank-Nicholson type:
\begin{align}
	&h_{j, n+1} 
	-
	\frac{\Delta \tau}{(j +\frac{1}{2})d_{n+\frac{1}{2}}\Delta \xi^2}
	\left\{
		jk_{j-\frac{1}{2},n+\frac{1}{2}} p_{j-1,n+1}
		-
		[
			jk_{j-\frac{1}{2},n+\frac{1}{2}}
			+
			(j+1)k_{j+\frac{1}{2},n+\frac{1}{2}}
		]p_{j,n+1}
		+
		(j+1)k_{j+\frac{1}{2}} p_{j+1,n+1}
	\right\}
	\nonumber \\
	&=h_{j,n}
	-
	\frac{ \Delta \tau}{(j +\frac{1}{2})d_{n+\frac{1}{2}}\Delta \xi^2}
	\left\{
		jk_{j-\frac{1}{2},n+\frac{1}{2}} p_{j-1,n}
		-
		[
			jk_{j-\frac{1}{2},n+\frac{1}{2}}
			+
			(j+1)k_{j+\frac{1}{2},n+\frac{1}{2}}
		]p_{j,n}
		+
		(j+1)k_{j+\frac{1}{2},n+\frac{1}{2}} p_{j+1,n}
	\right\}
\end{align}
We may re-express the Crank-Nicholson equation as a pentadiagonal system for $\vec{h}_{n+1}$:
\begin{equation}
	h_{j, n+1} 
	-
	\frac{2\Delta \tau}{(2j + 1)d_{n+\frac{1}{2}} \Delta \xi^2}
	\sum_{k=-2}^2 a_{j+k,n+\frac{1}{2}} h_{j+k,n+1}
	=
	h_{j, n} 
	+
	\frac{2 \Delta \tau}{(2j +1)d_{n+\frac{1}{2}} \Delta \xi^2}
	\sum_{k=-2}^2 a_{j+k,n+\frac{1}{2}} h_{j+k,n}
\end{equation}
where the coefficients are given by
%\begin{align}
%	a_{j-2,n+\frac{1}{2}}
%	&=
%	-\left[ \frac{j(j-1)}{j - \frac{1}{2}} \frac{\lambda^2}{r_{n+\frac{1}{2}}^2 \Delta \xi^2} \right] k_{j-\frac{1}{2},n+\frac{1}{2}}
%	\\
%	a_{j-1,n+\frac{1}{2}}
%	&=
%	j \left[
%		1 + \left(
%			\frac{j}{j + \frac{1}{2}} + \frac{2j - 1}{j - \frac{1}{2}}
%		\right)
%		\frac{\lambda^2}{r_{n+\frac{1}{2}}^2 \Delta \xi^2}
%	\right] k_{j-\frac{1}{2},n+\frac{1}{2}}
%	+
%	\left[\frac{j(j+1)}{j + \frac{1}{2}}\frac{\lambda^2}{r_{n+\frac{1}{2}}^2 \Delta \xi^2} \right] k_{j+\frac{1}{2},n+\frac{1}{2}}
%	\\
%	a_{j,n+\frac{1}{2}}
%	&=
%	- j \left[
%		1 + \left(
%			\frac{j}{j - \frac{1}{2}} + \frac{2j +1}{j+\frac{1}{2}}
%		\right) 
%		\frac{\lambda^2}{r_{n+\frac{1}{2}}^2 \Delta \xi^2}
%	\right]
%	k_{j-\frac{1}{2},n+\frac{1}{2}}
%	\nonumber \\ &\hphantom{=}\qquad
%	-
%	(j+1) \left[
%		1 + \left(
%			\frac{j + 1}{j + \frac{3}{2}}+ \frac{2j +1}{j+\frac{1}{2}}
%		\right)\frac{\lambda^2}{r_{n+\frac{1}{2}}^2 \Delta \xi^2} 
%	\right]k_{j+\frac{1}{2},n+\frac{1}{2}}
%	\\
%	a_{j+1,n+\frac{1}{2}}
%	&=
%	\left[
%		\frac{j (j + 1)}{j + \frac{1}{2}} \frac{\lambda^2}{r_{n+\frac{1}{2}}^2 \Delta \xi^2}
%	\right] k_{j-\frac{1}{2},n+\frac{1}{2}}
%	+ (j+1)
%	\left[
%		1 + \left(
%			\frac{j+1}{j + \frac{1}{2}} + \frac{2j + 3}{j + \frac{3}{2}}
%		\right)\frac{\lambda^2}{r_{n+\frac{1}{2}}^2 \Delta \xi^2}
%	\right]k_{j+\frac{1}{2},n+\frac{1}{2}}
%	\\
%	a_{j+2,n+\frac{1}{2}}
%	&=
%	-
%	\left[
%		\frac{(j+1)(j+2)}{j +\frac{3}{2}} \frac{\lambda^2}{r_{n+\frac{1}{2}}^2 \Delta \xi^2}
%	\right]k_{j+\frac{1}{2},n+\frac{1}{2}}
%\end{align}
%Actually, these coefficients may be rewritten as
\begin{align}
	a_{j-2,n+\frac{1}{2}}
	&=
	-\left[ \frac{j-1}{2j - 1} \frac{2\lambda^2}{r_{n+\frac{1}{2}}^2 \Delta \xi^2} \right] j k_{j-\frac{1}{2},n+\frac{1}{2}}
	\\
	a_{j-1,n+\frac{1}{2}}
	&=
	\left[
		1 + \frac{3j+1}{2j + 1}
		\frac{2\lambda^2}{r_{n+\frac{1}{2}}^2 \Delta \xi^2}
	\right] jk_{j-\frac{1}{2},n+\frac{1}{2}}
	+
	\left[
		\frac{j}{2j + 1}\frac{2\lambda^2}{r_{n+\frac{1}{2}}^2 \Delta \xi^2}
	\right] (j+1) k_{j+\frac{1}{2},n+\frac{1}{2}}
	\\
	a_{j,n+\frac{1}{2}}
	&=
	-  \left[
		1 + \frac{3j - 1}{2j - 1}
		\frac{2\lambda^2}{r_{n+\frac{1}{2}}^2 \Delta \xi^2}
	\right]
	j k_{j-\frac{1}{2},n+\frac{1}{2}}
	-
	\left[
		1 + \frac{3j + 4}{2j + 3} \frac{2\lambda^2}{r_{n+\frac{1}{2}}^2 \Delta \xi^2} 
	\right](j+1) k_{j+\frac{1}{2},n+\frac{1}{2}}
	\\
	a_{j+1,n+\frac{1}{2}}
	&=
	\left[
		\frac{j + 1}{2j + 1} \frac{2 \lambda^2}{r_{n+\frac{1}{2}}^2 \Delta \xi^2}
	\right] jk_{j-\frac{1}{2},n+\frac{1}{2}}
	+
	\left[
		1 + \frac{3j+2}{2j + 1} \frac{2\lambda^2}{r_{n+\frac{1}{2}}^2 \Delta \xi^2}
	\right] (j+1)k_{j+\frac{1}{2},n+\frac{1}{2}}
	\\
	a_{j+2,n+\frac{1}{2}}
	&=
	-
	\left[
		\frac{j+2}{2j +3} \frac{2\lambda^2}{r_{n+\frac{1}{2}}^2 \Delta \xi^2}
	\right](j+1)k_{j+\frac{1}{2},n+\frac{1}{2}}
\end{align}
(derivation  commented out; doesn't hurt to double check these coefficients).

Note that if $\lambda = 0$ (no surface tension), then the problem reduces to a tridiagonal system.


\subsection{Difference equations near the boundaries}

Special attention needs to be given to the equations near the center and rim of the drop.


\subsubsection{$j=0$}

We have $h_{-1}(\tau) = h_{0}(\tau)$ and $p_{-1}(\tau) = p_0(\tau)$. We can simply set $k_{j-\frac{1}{2},n+\frac{1}{2}} = 0$  in order to satisfy the second condition, then combine $a_{j-1}$ and $a_j$ to satisfy the second condition.


\subsubsection{$j=1$}

We have $h_{-1}(\tau) = h_{0}(\tau)$. Simply combine $a_{j-2}$ and $a_{j-1}$ to satisfy this condition.

\subsubsection{$j=J-3$}

We have $h_{J-1}(\tau) = h^* - \alpha r_n \Delta \xi / 2$. Simply move the term multiplying $a_{j+2}$ to the right-hand side of the equation.


\subsubsection{$j=J-2$}


We have $h_{J-1}(\tau) = h^* - \alpha r_n \Delta \xi / 2$ and $h_{J}(\tau) = h^* + \alpha r_n \Delta \xi / 2$. Simply move the terms multiplying $a_{j+1}$ and $a_{j+2}$ to the right-hand side of the equation.


\pagebreak

\section{Algorithm}

Let's keep $t$ as is, only redefine $\sigma \rightarrow \xi$. Thus, replace $d_{n+\frac{1}{2}}$ with $r_{n+\frac{1}{2}}^2$ in the above equations.

\begin{enumerate}
	\item (Initialization) Set $n = 0$ and $\vec{h}_n = \vec{h}_0$ (initial condition).
	\item (Prediction)
		Given $\vec{h}_n$, $\vec{\nu} = [\mu / (\rho g), \lambda, h^*, \alpha, \mit{\Omega}]$, $\Delta t$, $\Delta \xi$. Set $\Delta t' = \Delta t/2$, $\vec{h}'= \vec{h}_n$. Compute the drop radius,
		\[
			r = 
			\left[
	 	\frac{\mit{\Omega}}{\pi  \Delta \xi^2 \left[ Jh^* +   \sum_{j=1}^{J-1} j (h'_{j-1} + h'_{j}) \right]}
	 \right]^{\frac{1}{2}}
		\]
		the coefficients,
		\[
			T = 
			\frac{2\Delta t'}{r^2\Delta \xi^2},
			\qquad
			S = 
			\frac{2\lambda^2}{r^2\Delta \xi^2},
			\qquad
			R
			=
			\frac{\alpha r \Delta \xi}{2}
		\]
		and, finally, the diffusion coefficients,
		\[
			k_{j+\frac{1}{2}} = \frac{\rho g}{3 \mu} \left( 
				\frac{h'_{j} + h'_{j+1}}{2}
			\right)^3,
			\qquad
			k_{j-\frac{1}{2}} = \frac{\rho g}{3 \mu} \left( 
				\frac{h'_{j-1} + h'_{j}}{2}
			\right)^3,
			\qquad
			j = 0, 1, \dots, J-2
		\]
		Project $\vec{h}$ to $t_{n+\frac{1}{2}}$ using a centered Taylor series projection. That is, set $\vec{v} = \vec{h}_n$ and solve the pentadiagonal system,
		\[
			u_j
			-
			\frac{T}{2 j + 1}
			\sum_{k=-2}^2 a_{j+k} u_{j+k}
			=
			v_{j} 
			+
			\frac{T}{2 j + 1}
			\left(
			f_j
			+
			\sum_{k=-2}^2 a_{j+k} v_{j+k}
			\right)
		\]
		 where the coefficients $a_{j+k}$ and $f_j$ are given below. Then, set $\vec{h}_{n+\frac{1}{2}} = \vec{u}$.
	\item (Correction) Now set $\Delta t' = \Delta t$ and $\vec{h}'= \vec{h}_{n+\frac{1}{2}}$. Recompute $r$, $T$, $S$, $R$, and $k$ as defined above using these new parameters. Then, solve the pentadiagonal system above with $\vec{v} = \vec{h}_n$. Set $\vec{h}_{n+1} = \vec{u}$. 
	\item (Recursion) Advance $n$ by 1 and repeat steps 2 and 3.
\end{enumerate}

\subsection{Pentadiagonal coefficients}

\subsubsection{$2 \le j \le J-4$}
\begin{align}
	a_{j-2}
	&=
	-S\left( \frac{j-1}{2j - 1} \right) j k_{j-\frac{1}{2}}
	\\
	a_{j-1}
	&=
	\left[
		1 + S \left(\frac{3j+1}{2j + 1}\right)
	\right] jk_{j-\frac{1}{2}}
	+
	S \left(
		\frac{j}{2j + 1}
	\right) (j+1) k_{j+\frac{1}{2}}
	\\
	a_{j}
	&=
	-  \left[
		1 + S \left( \frac{3j - 1}{2j - 1} \right)
	\right]
	j k_{j-\frac{1}{2}}
	-
	\left[
		1 + S\left( \frac{3j + 4}{2j + 3} \right)
	\right](j+1) k_{j+\frac{1}{2}}
	\\
	a_{j+1}
	&=
	S\left(
		\frac{j + 1}{2j + 1} 
	\right) jk_{j-\frac{1}{2}}
	+
	\left[
		1 + S \left( \frac{3j+2}{2j + 1} \right)
	\right] (j+1)k_{j+\frac{1}{2}}
	\\
	a_{j+2}
	&=
	-S
	\left(
		\frac{j+2}{2j +3} 
	\right)(j+1)k_{j+\frac{1}{2}}
	\\
	f_j 
	&=
	0
\end{align}

\subsubsection{$j = 0$}
\begin{align}
	a_{j-2}
	&=
	0
	\\
	a_{j-1}
	&=
	0
	\\
	a_{j}
	&=
	-
	\left[
		1 + S\left( \frac{3j + 4}{2j + 3} - \frac{j}{2j + 1} \right)
	\right](j+1) k_{j+\frac{1}{2}}
	\\
	a_{j+1}
	&=
	\left[
		1 + S \left( \frac{3j+2}{2j + 1} \right)
	\right] (j+1)k_{j+\frac{1}{2}}
	\\
	a_{j+2}
	&=
	-S
	\left(
		\frac{j+2}{2j +3} 
	\right)(j+1)k_{j+\frac{1}{2}}
	\\
	f_j 
	&=
	0
\end{align}

\subsubsection{$j = 1$}

\begin{align}
	a_{j-2}
	&=
	0
	\\
	a_{j-1}
	&=
	\left[
		1 + S \left(\frac{3j+1}{2j + 1} -  \frac{j-1}{2j - 1}\right)
	\right] jk_{j-\frac{1}{2}}
	+
	S \left(
		\frac{j}{2j + 1}
	\right) (j+1) k_{j+\frac{1}{2}}
	\\
	a_{j}
	&=
	-  \left[
		1 + S \left( \frac{3j - 1}{2j - 1} \right)
	\right]
	j k_{j-\frac{1}{2}}
	-
	\left[
		1 + S\left( \frac{3j + 4}{2j + 3} \right)
	\right](j+1) k_{j+\frac{1}{2}}
	\\
	a_{j+1}
	&=
	S\left(
		\frac{j + 1}{2j + 1} 
	\right) jk_{j-\frac{1}{2}}
	+
	\left[
		1 + S \left( \frac{3j+2}{2j + 1} \right)
	\right] (j+1)k_{j+\frac{1}{2}}
	\\
	a_{j+2}
	&=
	-S
	\left(
		\frac{j+2}{2j +3} 
	\right)(j+1)k_{j+\frac{1}{2}}
	\\
	f_j 
	&=
	0
\end{align}



\subsubsection{$j = J-3$}

\begin{align}
	a_{j-2}
	&=
	-S\left( \frac{j-1}{2j - 1} \right) j k_{j-\frac{1}{2}}
	\\
	a_{j-1}
	&=
	\left[
		1 + S \left(\frac{3j+1}{2j + 1}\right)
	\right] jk_{j-\frac{1}{2}}
	+
	S \left(
		\frac{j}{2j + 1}
	\right) (j+1) k_{j+\frac{1}{2}}
	\\
	a_{j}
	&=
	-  \left[
		1 + S \left( \frac{3j - 1}{2j - 1} \right)
	\right]
	j k_{j-\frac{1}{2}}
	-
	\left[
		1 + S\left( \frac{3j + 4}{2j + 3} \right)
	\right](j+1) k_{j+\frac{1}{2}}
	\\
	a_{j+1}
	&=
	S\left(
		\frac{j + 1}{2j + 1} 
	\right) jk_{j-\frac{1}{2}}
	+
	\left[
		1 + S \left( \frac{3j+2}{2j + 1} \right)
	\right] (j+1)k_{j+\frac{1}{2}}
	\\
	a_{j+2}
	&=
	0
	\\
	f_j 
	&=
	-2\left(
		h^* + R
	\right)
	S
	\left(
		\frac{j+2}{2j +3} 
	\right)(j+1)k_{j+\frac{1}{2}}
\end{align}


\subsubsection{$j = J-2$}

\begin{align}
	a_{j-2}
	&=
	-S\left( \frac{j-1}{2j - 1} \right) j k_{j-\frac{1}{2}}
	\\
	a_{j-1}
	&=
	\left[
		1 + S \left(\frac{3j+1}{2j + 1}\right)
	\right] jk_{j-\frac{1}{2}}
	+
	S \left(
		\frac{j}{2j + 1}
	\right) (j+1) k_{j+\frac{1}{2}}
	\\
	a_{j}
	&=
	-  \left[
		1 + S \left( \frac{3j - 1}{2j - 1} \right)
	\right]
	j k_{j-\frac{1}{2}}
	-
	\left[
		1 + S\left( \frac{3j + 4}{2j + 3} \right)
	\right](j+1) k_{j+\frac{1}{2}}
	\\
	a_{j+1}
	&=
	0
	\\
	a_{j+2}
	&=
	0
	\\
	f_j 
	&=
	-2\left(
		h^* - R
	\right)
	S
	\left(
		\frac{j+2}{2j +3} 
	\right)(j+1)k_{j+\frac{1}{2}}
	\nonumber \\ & \hphantom{=}+
	2 (h^* + R)
	\left\{
		S\left(
		\frac{j + 1}{2j + 1} 
	\right) jk_{j-\frac{1}{2}}
	+
	\left[
		1 + S \left( \frac{3j+2}{2j + 1} \right)
	\right] (j+1)k_{j+\frac{1}{2}}
	\right\}
\end{align}



\end{document}